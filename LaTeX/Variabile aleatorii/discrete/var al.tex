\documentclass{article}
\usepackage{amsmath}
\usepackage{amsfonts}

\begin{document}
Variabile discrete:
\[
  \begin{pmatrix}
    x_1 & x_2 & x_3 & \ldots & x_n \\
    p_1 & p_2 & p_3 & \ldots & p_n \\
  \end{pmatrix}
\]

Operatii posibile:
\begin{description}
  \item[Adunare] $\rightarrow$ Se aduna toate valorile, se trec in tabel o singura data, crescator si probabilitatile se calculeaza pe baza definitiei.
  \item[Inmultire] $\rightarrow$ Se inmultesc toate valorile, se trec in tabel o singura data, crescator si probabilitatile se calculeaza pe baza definitiei.
  \item[Inmultire cu constanta] $\rightarrow$ Se inmulteste constanta cu fiecare valoare, se trec in tabel o singura data, crescator, si probabilitatile raman la fel.
  \item[Ridicare la putere] $\rightarrow$ Se ridica fiecare valoare la putere, se trec in tabel o singura data, crescator, si probabilitatile raman la fel.
\end{description}

Formule cu variabile discrete:\\
\begin{tabular}{lll}
  $M(X)$      & = & $ x_1 \times p_1 + x_2 \times p_2 + \ldots + x_n \times p_n $ \\
  $D^2(X)$    & = & $ M(X^2) - [M(X)]^2 $                                         \\
  $\sigma(X)$ & = & $ \sqrt[]{D^2(X)} $                                           \\
  $F_X(x)$    & = & $ P(X < x) $                                                  \\
\end{tabular}
\end{document}
